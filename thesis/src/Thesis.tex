\documentclass[11pt]{report}

% ------------------------------------------------------------------------------
% Dependencies

%\usepackage{cmbright}     % Computer Modern Bright
%\usepackage{eucal}        % Euler (calligraphic/script)
\usepackage{bm}            % Bold math font
\usepackage[at]{easylist}  % Easy-to-use lists
\usepackage{amsmath}       % The AMS math package
\usepackage{amssymb}       % The AMS symbols package
\usepackage{mathtools}     % Extensions to `amsmath'
\usepackage{wasysym}       % Various symbols, including smiley faces
\usepackage{stmaryrd}      % Logic and CS symbols
\usepackage{xfrac}         % Split level fractions with \sfrac
\usepackage{todonotes}     % TODO notes
\usepackage{enumitem}      % Customize `enumerate' lists
\usepackage{bussproofs}    % Gentzen-style inference rules
\usepackage{fontspec}      % FIXME
\usepackage{listings}      % Code listings
\usepackage{geometry}
\usepackage{hyperref}
\usepackage{cite}
\usepackage[numbib]{tocbibind}
\usepackage{titlesec}
\usepackage{blindtext}
\usepackage{color}
\usepackage{geometry}

% ------------------------------------------------------------------------------
% Chapter titles

\definecolor{gray75}{gray}{0.75}
\titleformat{\chapter}[hang]{\Huge\bfseries}{%
  \thechapter\hspace{20pt}\textcolor{gray75}{|}\hspace{20pt}%
}{0pt}{\Huge\bfseries}
\titlespacing*{\chapter}{0pt}{-20pt}{40pt}

% ------------------------------------------------------------------------------
% Code blocks

\usepackage[outputdir=../out]{minted}

\setmonofont{DejaVu Sans Mono}[Scale=MatchLowercase]

% ------------------------------------------------------------------------------
% Hyperlinks

\definecolor{gray75}{gray}{0.75}

\hypersetup{
  colorlinks  = true,
  linkcolor   = blue,
  urlcolor    = blue,
  citecolor   = blue,
  anchorcolor = blue,
}

% ------------------------------------------------------------------------------
% Page geometry

\geometry{
  hmargin        = 1.250in,
  vmargin        = 1.000in,
  marginparwidth = 0.750in,
  marginparsep   = 0.125in,
  heightrounded  = true,
}

% ------------------------------------------------------------------------------
% TikZ

\usepackage{tikz}
\usetikzlibrary{arrows,shapes,snakes,automata,backgrounds,petri,positioning}

% ------------------------------------------------------------------------------
% Short names for fonts

\newcommand{\textbs}[1]{{\sffamily\fontseries{sbc}\selectfont #1}}

\newcommand{\mathbs}[1]{\ensuremath{\text{\textbs{#1}}}}
\renewcommand{\mathtt}[1]{\ensuremath{\texttt{#1}}}

\newcommand{\mrs}[1]{\ensuremath{\mathnormal{#1}}} % Reset font to normal
\newcommand{\mbf}[1]{\ensuremath{\mathbf{#1}}}     % Boldface
\newcommand{\mbs}[1]{\ensuremath{\mathbs{#1}}}     % Bold + sans-serif
\newcommand{\mbb}[1]{\ensuremath{\mathbb{#1}}}     % Blackboard bold
\newcommand{\mtt}[1]{\ensuremath{\mathtt{#1}}}     % Teletype
\newcommand{\mrm}[1]{\ensuremath{\mathrm{#1}}}     % Serif ("roman")
\newcommand{\msf}[1]{\ensuremath{\mathsf{#1}}}     % Sans-serif
\newcommand{\msc}[1]{\ensuremath{\mathsc{#1}}}     % Small-caps
\newcommand{\mcl}[1]{\ensuremath{\mathcal{#1}}}    % Calligraphic
\newcommand{\msr}[1]{\ensuremath{\mathscr{#1}}}    % Script
\newcommand{\mfr}[1]{\ensuremath{\mathfrak{#1}}}   % Fraktur

% ------------------------------------------------------------------------------
% Various kinds of brackets

\makeatletter
\DeclareFontFamily{OMX}{MnSymbolE}{}
\DeclareSymbolFont{MnLargeSymbols}{OMX}{MnSymbolE}{m}{n}
\SetSymbolFont{MnLargeSymbols}{bold}{OMX}{MnSymbolE}{b}{n}
\DeclareFontShape{OMX}{MnSymbolE}{m}{n}{
    <-6>  MnSymbolE5
   <6-7>  MnSymbolE6
   <7-8>  MnSymbolE7
   <8-9>  MnSymbolE8
   <9-10> MnSymbolE9
  <10-12> MnSymbolE10
  <12->   MnSymbolE12
}{}
\DeclareFontShape{OMX}{MnSymbolE}{b}{n}{
    <-6>  MnSymbolE-Bold5
   <6-7>  MnSymbolE-Bold6
   <7-8>  MnSymbolE-Bold7
   <8-9>  MnSymbolE-Bold8
   <9-10> MnSymbolE-Bold9
  <10-12> MnSymbolE-Bold10
  <12->   MnSymbolE-Bold12
}{}

\let\llangle\@undefined
\let\rrangle\@undefined
\DeclareMathDelimiter{\llangle}{\mathopen}{MnLargeSymbols}{'164}{MnLargeSymbols}{'164}
\DeclareMathDelimiter{\rrangle}{\mathclose}{MnLargeSymbols}{'171}{MnLargeSymbols}{'171}
\makeatother

\newcommand{\sbkt}[2][]{\ensuremath{{#1\llbracket{}} {#2} {#1\rrbracket{}}}}
\newcommand{\abkt}[2][]{\ensuremath{{#1\langle{}} {#2} {#1\rangle{}}}}
\newcommand{\aabkt}[2][]{\ensuremath{\llangle[#1] {#2} \rrangle[#1]}}

% ------------------------------------------------------------------------------
% TODO notes

\newlength{\fixmewidth}
\setlength{\fixmewidth}{0.7\textwidth}
\newcommand{\fixme}[1]{%
  \begin{minipage}[c]{\fixmewidth}%
  \todo[color=green!40,inline]{\textsc{fixme:} #1}%
  \end{minipage}}
\newcommand{\sfixme}[0]{%
  \begin{minipage}[c]{3.5em}%
  \todo[color=green!40,inline]{\textsc{fixme}}%
  \end{minipage}}

% ------------------------------------------------------------------------------
% Miscellaneous other stuff

\newcommand{\eps}[0]{\varepsilon}

\newcommand{\catop}[1]{\ensuremath{{#1}^{\msf{op}}}}

\newcommand{\comp}[0]{\circ}
\newcommand{\tens}[0]{\otimes}

\newcommand{\example}[0]{\mathrm{example}}

\newcommand{\wrap}[1]{#1}

\DeclareMathOperator{\image}{im}
\DeclareMathOperator{\domain}{dom}

\newcommand{\aside}[1]{\hfill{} ({#1})}
\newcommand{\forceNewLine}[0]{{\hspace{1em}\newline{}}}
\renewcommand{\emptyset}[0]{\varnothing}

\renewcommand{\cdots}[0]{\makebox[1em][c]{${\cdot}$\hfil${\cdot}$\hfil${\cdot}$}}
\renewcommand{\dotsc}[0]{\makebox[1em][c]{.\hfil.\hfil.}}

\newlength{\stextwidth}
\newcommand{\makesamewidth}[3][c]{%
  \settowidth{\stextwidth}{#2}%
  \makebox[\stextwidth][#1]{#3}}

\newcommand{\email}[1]{\href{mailto:#1}{\texttt{#1}}}
\renewcommand{\thefootnote}{[\roman{footnote}]}

% ------------------------------------------------------------------------------

% ==============================================================================
% ==============================================================================
% ==============================================================================

\begin{document}

\title{A Generic Implementation of Equality Saturation in Haskell}
\author{%
  Remy Goldschmidt \\
  \email{regolds2@illinois.edu} \\
  University of Illinois at Urbana-Champaign
}

\maketitle

\tableofcontents

%%%%%%%%%%%%%%%%%%%%%%%%%%%%%%%%%%%%%%%%%%%%%%%%%%%%%%%%%%%%%%%%%%%%%%%%%%%%%%%%

\chapter{Introduction}

Equality saturation is a framework for optimization first introduced in
a 2009 POPL paper~\cite{tate-2009} by Tate et al.
The \textit{phase ordering problem} in compiler optimization is essentially
the issue of figuring out in what order optimizations should be applied to code;
this problem is very difficult because some optimizations expose code that
allows other optimizations to be applied, while other optimizations remove
opportunities to apply optimizations. Equality saturation solves the phase
ordering problem by restructuring optimization as saturation-based automated
theorem proving (``forward chaining'') followed by combinatorial optimization.

The basic outline of equality saturation is that the user must first convert
a piece of code (usually a control flow graph) into a referentially transparent
directed graph with sharing, which is called a \textit{program expression graph}
(PEG)\footnotemark. Here, ``referentially transparent'' means that, assuming
there is a transition system $(S, {\to})$ representing the semantics of the
language, the semantics of a PEG node are defined purely by the node label and
the semantics of the children of that node (its ``out-neighbors''). Once we have
a PEG, equality saturation needs two other things from the user to optimize it:

\footnotetext{
  Note that the way PEGs are defined in \cite{tate-2009} is actually more
  specific (due to the details of optimizing imperative languages) than the
  way I will use the term in this paper; this is one of the ways in which
  this description of equality saturation is more generic than previous
  expositions.
}

\begin{easylist}[enumerate]
@ {%
  A term-rewriting system on PEGs, whose rules define the basic optimizations
  that the equality saturation engine will compose together. For example, this
  term-rewriting system could simply be the operational semantics TRS of the
  programming language, in terms of PEGs.
}
@ {%
  A heuristic for the runtime performance of a given PEG. Since a PEG is
  referentially transparent, this can be expressed inductively; e.g.: the
  type of a heuristic function of a PEG whose node label set is $L$ can be
  $(L \to R) \times (\mbb{P}(R) \to R)$\footnotemark{} rather than
  $\mtt{PEG}_L \to R$ without losing any expressiveness.

  \footnotetext{
    $R$ here is an arbitrary ordered semiring; usually this will be an integer
    or floating-point type.
  }
}
\end{easylist}


 and a heuristic for
the performance of a

Then
this PEG is given to the equality saturation algorithm along with a set of
rewrite rules on the term language given by PEG nodes
We then take
as input from the user

The equality saturation algorithm, as originally described, is comprised of the
following steps:

\begin{easylist}[enumerate]
@ Convert the input CFG to a PEG
\end{easylist}

% \begin{easylist}[itemize]
% @ Equality saturation
% @@ \cite{tate-2009}
% @@ \cite{tate-2012-eqsat}
% @@ \cite{stepp-2011}
% @ Proof generalization
% @@ \cite{tate-2012-proofgen}
% @ Term indexing
% @@ \cite{handbook-ch26}
% @@ \cite{graf-1994}
% @@ \cite{holen-2013}
% @ Unification / Matching
% @@ \cite{baader-2015}
% @@ \cite{plotkin-1972}
% @@ \cite{eker-2003}
% @@ \cite{eker-1995}
% @@ \cite{forgy-1979}
% @@ \cite{doorenbos-2001}
% @ Resource Estimation
% @@ \cite{carbonneaux-2015}
% @@ \cite{hoffmann-2012}
% @ Static Analysis
% @@ \cite{calcagno-2011}
% @@ \cite{brotherston-2017}
% @ SMT solving
% @@ \cite{bjorner-2015}
% @ Reference
% @@ \cite{baader-1998}
% @ Miscellaneous
% @@ \cite{alpuente-2014}
% @@ \cite{lewenstein-2013}
% @@ \cite{stampoulis-2010}
% @@ \cite{sjoberg-2014}
% @@ \cite{bachmair-2003}
% @@ \cite{payet-2008}
% @@ \cite{baumgartner-2007}
% @@ \cite{darais-2017}
% \end{easylist}

%%%%%%%%%%%%%%%%%%%%%%%%%%%%%%%%%%%%%%%%%%%%%%%%%%%%%%%%%%%%%%%%%%%%%%%%%%%%%%%%

\bibliography{sources}
\bibliographystyle{plainurl}

%%%%%%%%%%%%%%%%%%%%%%%%%%%%%%%%%%%%%%%%%%%%%%%%%%%%%%%%%%%%%%%%%%%%%%%%%%%%%%%%

\end{document}
